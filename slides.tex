\documentclass{beamer}

\usepackage[utf8]{inputenc}
\usepackage[frenchb]{babel}
\usepackage{verbatim}
\usepackage{graphicx}
\usepackage{color}
\usepackage{hyperref}
\usepackage{url}
\usepackage{moreverb}
\usepackage{fancyvrb}
\usepackage{natbib}
\usepackage{eulervm}
\usepackage{auto-pst-pdf}
\usepackage{pst-plot}
\usepackage{multirow}


\hypersetup{colorlinks=true, linkcolor=black, urlcolor=blue}
\usetheme{boxes}
\beamertemplatenavigationsymbolsempty
\setbeamertemplate{sections/subsections in toc}[circle]
\setbeamertemplate{footline}[frame number]
\setbeamertemplate{itemize items}[circle]
\setbeamertemplate{itemize subitem}[square]

\title{{\bf Scikit-Learn in particle physics}}
\author{Gilles Louppe}
\institute{CERN, Switzerland}
\date{November 18, 2014}

\newcommand{\todo}[1]{\textcolor{red}{[TODO] #1}}

\definecolor{lightgreen}{rgb}{0.0,0.8,0.0}
\definecolor{lightblue}{rgb}{0.3,0.8,1.0}
\definecolor{lightred}{rgb}{0.874,0.180,0.105}
\definecolor{gray}{rgb}{0.4,0.4,0.4}
\definecolor{lightgray}{rgb}{0.8,0.8,0.8}
\definecolor{shadecolor}{rgb}{0.9,0.9,0.9}
\newrgbcolor{mygreen}{.00 .5 .00}
\newrgbcolor{myyellow}{.6 .6 .00}

\DeclareMathOperator*{\argmax}{arg\,max}


\newrgbcolor{mygreen}{.00 .5 .00}
\newcommand{\X}[1]{\textcolor{blue}{#1}}
\newcommand{\y}[1]{\textcolor{red}{#1}}
\newcommand{\model}[1]{\textcolor{mygreen}{#1}}
\newcommand{\loss}[1]{\textcolor{lightblue}{#1}}
\setbeamertemplate{enumerate items}[circle]

\begin{document}

\renewcommand{\inserttotalframenumber}{XX}

% -----------------------------------------------------------------------------

\begin{frame}
\titlepage
\end{frame}

% -----------------------------------------------------------------------------

{
\usebackgroundtemplate{\includegraphics[height=\paperheight]{./figures/lhc.jpg}}
\setbeamercolor{frametitle}{fg=white}
\setbeamercolor{normal text}{fg=white}

\begin{frame}{High Energy Physics (HEP)}

\begin{columns}
    \begin{column}{0.5\textwidth}
    \onslide<3->{
    \begin{figure}
    \includegraphics[width=0.8\textwidth]{./figures/atlas.jpg}
    \end{figure}
    \begin{figure}
    \includegraphics[width=0.8\textwidth]{./figures/cms.jpg}
    \end{figure}}
    \end{column}
    \begin{column}{0.5\textwidth}
    \onslide<2->{
        \begin{center}
        \color{white} \Large \it  Study the nature of the constituents of matter
        \end{center}
    }
    \onslide<3->{
    \begin{figure}
    \includegraphics[width=0.8\textwidth]{./figures/tunnel.jpg}
    \end{figure}}
    \end{column}
\end{columns}
\end{frame}
}


% -----------------------------------------------------------------------------

{
%\usebackgroundtemplate{\includegraphics[height=\paperheight]{./figures/collision.jpg}}
\setbeamercolor{frametitle}{fg=white}
\setbeamercolor{normal text}{fg=white}
\setbeamercolor{background canvas}{bg=black}

\begin{frame}{Particle detector 101}
    \only<1>{
        \includegraphics[width=\textwidth]{./figures/collision.jpg}
    }
    \only<2>{
        \includegraphics[width=\textwidth]{./figures/detector.jpg}
    }
\end{frame}
}

% -----------------------------------------------------------------------------

\begin{frame}{Data analysis tasks in detectors}
\begin{enumerate}

\item Track finding\\
    {\scriptsize Reconstruction of particle trajectories from hits in detectors}

    \vspace{0.5cm}

\item Budgeted classification\\
    {\scriptsize Real-time classification of events in triggers}

    \vspace{0.5cm}

\item Classification of signal / background events\\
    {\scriptsize Offline statistical analysis for discovery of new particles}

    \vspace{0.5cm}

\item ...

\end{enumerate}
\end{frame}

% -----------------------------------------------------------------------------

\begin{frame}{The Higgs Boson challenge}
% description
% unique measure
% sample weights
\end{frame}

\begin{frame}{Winning methods}

\end{frame}

\begin{frame}{Lessons learned}
% enthousiasm towards unknown ML techniaues
% http://www.kaggle.com/c/higgs-boson/forums/t/10350/how-physicists-fared

% tMVA + ROOT, often considered as a black box
% simple statistical methods
% lack of adoption because understandable or because it works
% overall, the whole statistical and data ecosystem has been developed in vase clos
\end{frame}

\begin{frame}{What about Scikit-Learn?}
% sample weights
% no root format
\end{frame}

% -----------------------------------------------------------------------------

\begin{frame}{Conclusions}
% a lot of things to do, but things are changing! (slowly)
% mutual education and understanding
% go beyong good performance => use models to understqnd whats going on
% call for collaboration between data science and HEP, equal to equal
\end{frame}

% -----------------------------------------------------------------------------

\appendix

\begin{frame}
\begin{center}
{\Huge  Questions?}
\end{center}
\end{frame}

\end{document}
